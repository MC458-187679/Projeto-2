\section{Definições}

Um alpinista deve escalar uma parede representada por uma grade $n \times m$, em cada célula $(i, j)$ da grade tem um custo $C_{i, j} > 0$ associado. O alpinista deve começar na base, linha 1, e chegar no topo, linha $n$, tentando minizar o grau de periculosidade (custo total) do caminho. Quando ele se encontra na célula $(i, j)$, com $1 \leq i < n$ e $1 \leq j \leq m$, os únicos movimentos possíveis são para as células $(i+1, j-1)$, $(i+1, j)$ ou $(i+1, j+1)$, quando elas fazem parte da parede.

\subsection{Definições Adicionais}

\begin{definition}[Caminho até o Topo]
    Considere um caminho $P = \left(p_1, \ldots, p_k\right)$, sendo $k$ a quantidade de vértices no caminho. Então, $P$ é um \textit{caminho até o topo} se ele termina em alguma célula do topo da parede, ou seja, $p_k = (n, j)$ para algum $1 \leq j \leq m$.
\end{definition}

\begin{definition}[Risco]
    O grau de periculosidade ou risco $R(P)$ de um caminho $P = \left(p_1, \ldots, p_k\right)$ é dado pela soma dos custos de cada célula do caminho, isto é,
    \[
        R(P) = \sum_{1 \leq i \leq k} C_{p_i}
    \]
\end{definition}

\begin{definition}[Risco Mínimo]
    O risco mínimo $R^*_{i, j}$ de uma célula $(i, j)$ é o risco de um \textit{caminho ótimo} de $(i, j)$, ou seja, um caminho até o topo $P$ com menor custo total $R(P)$ dentre os caminhos partindo de $(i, j)$.
\end{definition}

\section{Solução}

\begin{theorem}[subestrutura ótima]
    Seja $k \geq 2$ e suponha um caminho $P = \left(p_1, p_2, \ldots, p_k\right)$ até o topo partindo de $p_1$ e com risco mínimo $R(P) = R^*_{p_1}$. Então, o subcaminho $\left(p_2, \ldots, p_k\right)$ tem o menor risco possível partindo de $p_2$.
\end{theorem}

\begin{proof}
    Seja $S = \left(p_2, \ldots, p_k\right)$ o subcaminho de $P$ mencionado e suponha um caminho até o topo $S' = \left(s_1, \ldots, s_l\right)$ partindo de $s_1 = p_2$, com risco $R(S') = R^*_{p_2}$ mínimo. Assim, temos o caminho $P' = \left(p_1, s_1, \ldots, s_l\right)$, com risco
    \[
        R(P') = C_{p_1} + \sum_{i = 1}^l C_{s_i} = C_{p_1} + R(S')
    \]

    Suponha, por contradição, que $R(S') < R(S)$. Então, temos que
    \begin{align*}
        R(P') &= C_{p_1} + R(S')  \\
            &< C_{p_1} + R(S) \\
            &= C_{p_1} + \sum_{i = 2}^{k} C_{p_i} \\
            &= R(P)
    \end{align*}
    Logo, $P$ não é o caminho partindo de $p_1$ com menor risco, o que contradiz a suposição inicial.

    ~

    Portanto, $R(S) \leq R(S') = R^*_{p_2}$, isto é, $S$ tem risco mínimo dentre os caminhos partindo de $p_2$.
\end{proof}

\begin{samepage}
    \subsection{Fórmula de Recorrência} \label{sec:formrec}
\end{samepage}

Podemos notar que o caminho das células do topo com risco mínimo devem conter apenas a própria célula, então, $R^*_{n, j} = C_{n, j}$  para $1 \leq j \leq m$. Agora, para as células que não estão no topo, vamos ter que qualquer caminho até o topo, inclusive um caminho ótimo $P = \left(p_1, \ldots, p_k\right)$, terá no mínimo 2 vértices, ou seja, $k \geq 2$. Portanto, pela subestrutura ótima, temos que $R^*_{p_1} = C_{p_1} + R^*_{p_2}$.

Como $p_1$ deve ser uma célula válida, então temos $1 \leq i < n$ e $1 \leq j \leq m$ tal que $p_1 = (i, j)$. No entanto, o alpinista só pode ir de $(i, j)$ para as células superiores esquerda, central ou direita, isto é, $p_2 = (i+1, j-1)$ ou $p_2 = (i+1, j)$ ou $p_2 = (i+1, j+1)$, dado que essas posições são válidas. No caso geral, em que $1 < j < m$, temos que
\[
    R^*_{i, j} = \min\left\{C_{i, j} + R^*_{i+1,j-1}, C_{i, j} + R^*_{i+1,j}, C_{i, j} + R^*_{i+1,j+1}\right\}
\]

Quando $j - 1 < 1$ ou $j + 1 > n$, podemos expandir a definição de risco mínimo para que $R^*_{i, j} = \infty$, mantendo a válidade do caso geral. Assim, temos a seguinte relação recorrente:
\begin{align*}
    R^*_{n, j} &= C_{n, j} && \text{para $1 \leq j \leq m$}\\
    R^*_{i, j} &= \infty && \text{para $1 \leq i \leq n$ e $j < 1$ ou $j > m$} \\
    R^*_{i, j} &= C_{i, j} + \min\left\{R^*_{i+1,j-1}, R^*_{i+1,j}, R^*_{i+1,j+1}\right\} && \text{para $1 \leq i < n$ e $1 \leq j \leq m$}
\end{align*}

Note que cada linha $i$ depende apenas da linha superior $i+1$, com exceção do caso base $i = n$, que não depende de subcaminhos. Então, podemos calcular toda a matriz de risco mínimo $R^*_{i, j}$ a partir do topo da parede, sem necessidade de recursão ou memorização.

O resultado final $R^*$ do algoritmo é apenas menor dos riscos dentre os caminhos partindo da base, isto é
\[
    R^* = \min_{1 \leq j \leq m}\left\{R^*_{1, j}\right\}
\]

\noindent\begin{minipage}[t]{0.45\textwidth}
    \section{Algoritmo} \label{sec:alg}

    O algoritmo foi baseado em programação dinâmica, seguindo a recorrência apresentada com pequenas modificações. Para evitar tratamento das bordas da matriz $R$ e acessos repetidos, as variáveis $\id{R-esq}$, $\id{R-cen}$ e $\id{R-esq}$ foram usadas como uma janela dos possíveis risco mínimos partindo de $(i, j)$, como mostra a \cref{fig:janela}.

    \vskip0.5em
    ~

    \begin{minipage}{\textwidth}
        \centering\captionsetup{type=figure}\capstart
        \begingroup
            \def\passo{0.5}%
\newcommand{\cell}[2][\vphantom{$\id{R}$}]{\shortstack{\small #1 \\ \tiny $#2$}}%
\def\chave#1{[#1]}
\begin{tikzpicture}
    \draw[step=\passo*2cm,color=gray] (-4*\passo,0*\passo) grid (4*\passo,4*\passo);
    \node at (-3*\passo,+3*\passo) {\cell[$\id{R-esq}$]{i+1,j-1}};
    \node at (-1*\passo,+3*\passo) {\cell[$\id{R-cen}$]{i+1,j}};
    \node at (+1*\passo,+3*\passo) {\cell[$\id{R-dir}$]{i+1,j+1}};
    \node at (+3*\passo,+3*\passo) {\cell{i+1,j+2}};
    \node at (-3*\passo,+1*\passo) {\cell{i,j-1}};
    \node at (-1*\passo,+1*\passo) {\cell[$C\chave{i}\chave{j}$]{i,j}};
    \node at (+1*\passo,+1*\passo) {\cell{i,j+1}};
    \node at (+3*\passo,+1*\passo) {\cell{i,j+2}};

    \draw [thick,->] (0*\passo, 0 *\passo - 0.5*\passo) -- (0*\passo, -4*\passo + 0.5*\passo) node[midway, right] {$j \gets j + 1$};

    \draw[step=\passo*2cm,color=gray] (-4*\passo,-8*\passo) grid (4*\passo,-4*\passo);
    \node at (-3*\passo,-5*\passo) {\cell{i+1,j-2}};
    \node at (-1*\passo,-5*\passo) {\cell[$\id{R-esq}$]{i+1,j-1}};
    \node at (+1*\passo,-5*\passo) {\cell[$\id{R-cen}$]{i+1,j}};
    \node at (+3*\passo,-5*\passo) {\cell[$\id{R-dir}$]{i+1,j+1}};
    \node at (-3*\passo,-7*\passo) {\cell{i,j-2}};
    \node at (-1*\passo,-7*\passo) {\cell{i,j-1}};
    \node at (+1*\passo,-7*\passo) {\cell[$C\chave{i}\chave{j}$]{i,j}};
    \node at (+3*\passo,-7*\passo) {\cell{i,j+1}};
\end{tikzpicture}
        \endgroup
        \caption{Janela de possíveis subcaminhos.}
        \label{fig:janela}

        ~
    \end{minipage}

    ~

    Desse modo, só é necessário um novo acesso para a variável $\id{R-dir}$. Na última posição da linha, no entanto, $(i+1, m + 1)$ não é uma célula válida. Então, podemos usar a definição de risco mínimo expandida da seção \nameref{sec:formrec}, tratando essa posição de forma especial.

    \setlength{\parindent}{1em}
    Note que o algoritmo assume $n$ e $m$ positivos.
\end{minipage}%
\hfill%
\begin{minipage}[t]{0.47\textwidth}
    \begin{codebox}
        \Procname{$\proc{Risco-Mínimo}(C, n, m)$}
        \li Seja $R[1 \twodots n][1 \twodots m]$ uma nova matriz.
        \li
        \li \kw{para} $j = 1$ \kw{até} $m$
            \Do
        \li     $R[n][j] \gets C[n][j]$
            \End
        \li
        \li \kw{para} $i = n - 1$ descendo \kw{até} $1$
            \Do
        \li     $\id{R-esq} \gets \infty$
        \li     $\id{R-cen} \gets \infty$
        \li     $\id{R-dir} \gets C[i][1]$
        \li
        \li     \kw{para} $j = 1$ \kw{até} $m - 1$
                \Do
        \li         $\id{R-esq} \gets \id{R-cen}$
        \li         $\id{R-cen} \gets \id{R-dir}$
        \li         $\id{R-dir} \gets R[i+1][j+1]$
        \li
        \li         $\id{R-min} \gets \min\left(\id{R-esq}, \id{R-cen}, \id{R-dir}\right)$
        \li         $R[i][j] \gets C[i][j] + \id{R-min}$
                \End
        \li
        \li     $\id{R-esq} \gets \id{R-cen}$
        \li     $\id{R-cen} \gets \id{R-dir}$
        \li     $\id{R-dir} \gets \infty$
        \li
        \li     $\id{R-min} \gets \min\left(\id{R-esq}, \id{R-cen}, \id{R-dir}\right)$
        \li     $R[i][m] \gets C[i][m] + \id{R-min}$
            \End
        \li
        \li $\id{R-min} \gets R[1][1]$
        \li \kw{para} $j = 2$ \kw{até} $m$
            \Do
        \li     $\id{R-min} \gets \min(\id{R-min},~R[1][j])$
            \End
        \li \kw{retorna} $\id{R-min}$
    \end{codebox}
\end{minipage}

\subsection{Implementação}

O programa foi implementado em linguagem C, onde as matrizes $C$ e $R$ foram representadas em um buffer sequencial com ordem \textit{row-major}. Nessa representação, os elementos de uma mesma linha são guardados sequencialmente de forma que as posições $(i, j)$ e $(i, j + 1)$ ficam próximas na memória, mas $(i, j)$ e $(i + 1, j)$ podem ficar distantes.

Além disso, pode-se notar dos casos de teste que $n \leq 100$, $m \leq 2200$ e $C_{i, j} \leq 99$. Então, o custo $C_{i, j}$ foi representado por um inteiro de 8 bits (\texttt{uint8\_t}), já que sempre $C_{i, j} < 100 < 2^8$. O risco também pode ser limitado por
\begin{align*}
    R^*_p &\leq \max_{1 \leq i \leq n \text{~e~} 1 \leq j \leq m} \left\{R^*_{i, j}\right\} \\
        &= \max_{1 \leq j \leq m} \left\{R^*_{1, j}\right\} \\
        &\leq \sum_{i = 1}^n \max_{1 \leq j \leq m}\left\{C_{i, j}\right\} \\
        &\leq \sum_{i = 1}^n 99 \\
        &= 99 n \\
        &\leq 9900 \\
        &< 2^{16}
\end{align*}

Então, o risco foi representado com um inteiro de 16 bits (\texttt{uint16\_t}). Por causa disso, o valor de infinito foi usado como \mintinline{c}{UINT16_MAX}, que é garantidamente maior ou igual a $2^{16}$. De qualquer forma, o código fonte foi pensado de forma a facilitar a alteração dessas restrições, com as macros \texttt{CUSTO\_WIDTH} e \texttt{RISCO\_WIDTH}.

\section{Complexidade}

Podemos assumir que as operações aritméticas e as de acesso tem ordem $\Theta(1)$. Logo, o tempo de execução $t_i$ de cada uma dessas operações $i$ pode ser limitado pelas constantes $a \leq t_i \leq b$. Assim, o tempo do algortimo apresentado na \hyperref[sec:alg]{seção anterior}, para $n$ e $m$ positivos, pode ser majorado por

\begin{align*}
    T(n, m) &\geq a + \sum_{j = 1}^m a + \sum_{i = 1}^{n - 1} \left( a + \sum_{j = 1}^{m - 1} a \right) + \sum_{j = 2}^m a \\
    &= a + m a + (n - 1) (1 + (m - 1) a) + (m - 1) a \\
    &= n m a + m a \\
    &\geq a \cdot n m
\end{align*}

Da mesma forma, temos que
\[
    T(n, m) ~\leq~ b + \sum_{j = 1}^m b + \sum_{i = 1}^{n - 1} \left( b + \sum_{j = 1}^{m - 1} b \right) + \sum_{j = 2}^m b ~\leq~ 4 b \cdot n m
\]

Logo, a complexidade de tempo do algoritmo é $T(n, m) \in \Theta(n m)$.

Para o espaço, podemos ver que o único armazenamento adicional é da matriz $R$, com dimensões $n \times m$. Como a função não faz chamadas recursivas, temos que
\[
    E(n, m) = n m + \Theta(1) = \Theta(n m)
\]

Assumindo que $m \in O(n)$, podemos afirmar então que a complexidade de tempo é $T(n) \in O\left(n^2\right)$ e de espaço também é $E(n) \in O\left(n^2\right)$.
