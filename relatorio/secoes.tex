\makeatletter

\usepackage{nameref, titlesec}
\usepackage[hidelinks]{hyperref}


% formatação das seções
\titleformat{\section}[runin]
    {\titlerule{}\vspace{1ex}\normalfont\Large\bfseries}{}{1em}{}[.]
\titleformat{\subsection}[runin]
    {\normalfont\large\bfseries}{}{1em}{}[)]

% marcadores de nome na seção
\newcommand{\cur@section}{ERRO}
\newcommand{\cur@subsection}{ERRO}

% seções que atualizam esses marcadores
\let\old@section\section
\renewcommand{\section}[1]{%
    \renewcommand{\cur@section}{#1}%
    \renewcommand{\cur@subsection}{}%
    \old@section{#1}%
}
\renewcommand{\appendix}[1]{%
    \renewcommand{\cur@section}{#1}%
    \renewcommand{\cur@subsection}{}%
    \old@section{Apêndice #1}%
}
\let\old@subsection\subsection
\renewcommand{\subsection}[1]{%
    \renewcommand{\cur@subsection}{.\uppercase{#1}}%
    \old@subsection{#1}%
}

% nome com seção e subseção
\newcommand{\cur@itemname}{\cur@section\cur@subsection}

% comandos de referenciar
\newcommand{\new@ref}[3]{
    \newcommand{#1}[2][]{%
        \ifstrempty{##1}{%
            \hyperref[{##2}]{#2{##2}#3}%
        }{%
            \hyperref[{##1}]{#2{##2}#3}%
        }%
    }
}
\new@ref{\boldref}{\textbf}{}
% \new@ref{\thmref}{\bfseries\ref*}{}
\newcommand{\thmref}[2][]{\hyperref[#2]{\bfseries{}#1\ref*{#2}}}
\new@ref{\itemref}{\bfseries\nameref*}{)}
\new@ref{\qref}{\bfseries\nameref*}{.}

% linha final da página ou seção
\newcommand{\docline}[1][\\]{%
    #1\noindent\rule{\textwidth}{0.4pt}%
    \pagebreak%
}

% pula uma linha
\def\skipline{\vskip\baselineskip}

% linha horizontal menor
\def\itemsep{
    \noindent\hfil\rule{0.5\textwidth}{.2pt}\hfil
    \vskip1em
}

\makeatother
